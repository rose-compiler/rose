\section{Introduction}

\fixme{Need to identify top 30-40 security flaws with NIST.}

   The primary purpose of security flaw seeding is to automate the testing of static analysis
tools designed to report security flaws in software.  
% By seeding bugs into real applications in many different ways, a tool's ability to detect
% a high percentage of known (seeded) bugs can be predictive of the tool's ability to detect
% unknown bugs.
Some details presented in this design document for such a tool are specific to capabilities
within the ROSE compiler infrastructure.  ROSE is a source-to-source compiler
infrastructure for the custom analysis and transformation of source code and binary 
executables. ROSE is an open source project at Lawrence Livermore National Laboratory, 
more information is available at \href{http://www.roseCompiler.org}{ROSE}.

   Our purpose is first to define a simple compiler based approach to
the construction of test suits to support the evaluation of static analysis
tools; and provide some immediate help in this area to our collaborators.
A second use of this technology is to evaluate the security of 
applications, specifically if all possible known security flaws can be introduced and 
a percentage identified by static analysis, then we could have some statistical
certainty that a similar percentage of the unknown security flaws in the 
same application have been identified (see subsection \ref{sec:NewResearchArea}).
% This is the basis fo the theory behind
% evaluations of applications with seeded security flaws.  This subject requires
% substantial exporation to be validated in practice and this work provides a basis
% for the supporting infrastructure for such research work.

   In principle, seeding of flaws could be applied to either source code or binary
executables, but we restrict our attention to the seeding of security flaws in source
code. Note that ROSE as an infrastructure might be especially appropriate to both of
these approaches, since it handles the analysis and transformation of both source code and binary
executables.  The seeding of security flaws could be applied more generally to arbitrary
bug seeding, but we restrict our focus to only the seeding of security flaws into
arbitrary applications.

   This initial document is to permit a focus on the design of
a translator built using ROSE for the source-to-source transformation
of arbitrary applications to include security flaws for the purpose
of testing arbitrary static analysis tools.  The issues are dominantly
language independent and it should be possible to generate source
code for input applications over a range of languages.  ROSE at present supports
analysis and transformation of C, C++, Fortran 2003 (and older versions of Fortran), 
OpenMP, UPC, and recent initial support for PHP; support focused {\em only} on code
generation could be added for Java, Ada, and other languages, but such additional
languages are generally not a target of our research work). Such support for only code
generation would be of value for automated generation of test suits across multiple
languages.

\subsection{Benefits}
  % Outline benefits of automated generation of test codes.
   The development of test suites for the evaluation of static analysis tools
is labor intensive if done by hand.  Attempts to automate this have defined general code
generators that generally lacked a compiler's deeper level understanding of 
language specific code construction issues (e.g. with respect to what is valid code,
and language specific details, e.g. template instantiation in C++).
Specifically, the problem of enumerating
the valid programs defined by a union of a language grammar and a grammar defined by 
a set of security flaws is a fundamental programming language and compiler construction
topic.  Further the evaluation of security flaws is significantly more complex in the
context of real programs because of the more complex control flows (if not compounded by
the compromises of handling the shear size of realistic applications, which can force 
compromizes in the program analysis that change the results of evaluating only small 
test codes).  It is likely
critical to the evaluation of many static analysis tools to supplement their evaluation on
small generated test codes with their evaluation on larger application codes; where
the control flows and alias analysis are by definition more realistic.  Finally, the
support for automated security flaw seeding is not just limited to evaluation of flaws but
can likely be extended to handle the evaluation of false positive rates in static analysis
tools as well (see section \ref{sec:FalsePositiveEvaluation}).

\subsection{A Statistical Basis for Software Assurance}
\label{sec:NewResearchArea}
    The evaluation of unknown security flaws in application may be able to
be predicted with some degree of statistical confidence.  An approach for this
is based on the ability to seed security flaws into a target application and 
evaluate the number and types of security flaws that could be identified
using state-of-the-art static (and maybe dynamic) analysis tools.  If, for example,
only half of the seeded bugs can be identified, then some statistical basis
can be derived for an estimation of the number of remaining security flaws in 
the application and as a result the degree of confidence that the application 
can be trusted.  Variations of this technique can include:
\begin{enumerate}
   \item Counting unknown security flaws against the potential of an application for such
         vulnerabilities (e.g. an application without any arrays may be less prone to
         buffer overflows, similarly a language that automate bounds checking is similarly
         less prone to buffer overflows). 

   \item Randomization in the seeding of security flaws may permit the same level of
         statistical confidence in the evaluation of a static analysis tool (or an
         application, see ref{}) but be significantly more efficient.  A degree of
         coverage of all types of seeding of each security flaw must be maintained.
         The randomization may also be a function of the percentage of vulnerabilities 
         for each specific security flaw in the target application.  It can be expected
         that the selection of a application to seed with security flaws may define a 
         bias in the testing.  This likely has advantages and disadvantages in practice.
 
\end{enumerate}

This is ultimately a more complex subject that dealt with in this
document and it seriously lacks data to back up this argument.  More that an argument
that this is now software could be evaluated in the future, we define a research
area for further exploration.  The point is that a basis for this work is the
infrastructure required to seed security flaws into applications and that 
this infrastructure could have immediate use as a basis for the development of
automatically generated test suites for the evaluation of static analysis tools
by others (e.g. NIST, CERT, etc.).


\subsection{A Trivial Example}

   An initial tutorial example of a translator that introduces a 
buffer overflow to an existing application is available in the ROSE Tutorial
(Chapter 41) at {\em http://www.roseCompiler.org}.  This example detects a
vulnerability first (any place where an array is indexed in a loop construct)
and then modifies the indexing to guarantee a buffer overflow for all
array accesses.  The example represents an extremely narrow range of
the ways in which this specific security flaw can be introduced in an
arbitrary input application code.  To provide a concrete example, the code 
before seeding is:
{\mySmallFontSize
\begin{verbatim}
void foobar()
   {
  // Static array declaration
     float array[10];
     array[4] = 0;
     for (int i=0; i < 5; i++)
        {
          array[i] = 0;
        }
   }
\end{verbatim}
}
And the code after translation (with the seeded security flaw) is:
{\mySmallFontSize
\begin{verbatim}
void foobar()
{
// Static array declaration
  float array[10UL];
  array[4] = (0);
  for (int i = 0; i < 5; i++) {
// *** NOTE Seeded Security Flaw: BufferOverFlowSecurityFlaw 
    array[i + 10UL] = (0);
  }
}
\end{verbatim}
}
Note that this example is from a more involved version of the bug seeding
work (than what is in the ROSE Tutorial) (an initial prototype version being 
built to explore bug seeding more throughly). This work is in the current 
svn repository in the new {\em projects/bugSeeding} directory. Note that is
focused on a single kind of buffer overflow (array indexing in loops) and
that the transformed code adds a comment mark it clearly as a seeded security 
flaw (an easy trick in ROSE).
