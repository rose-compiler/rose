\section*{Appendix: Implementation Notes}

%\subsection{Implementation Notes}
Notes on the design/implementation of a translator to seed security flaws:
\begin{enumerate}

   \item To support an implementation, it may be more useful to define a grammar for
security flaws and in so doing define how it is nested in a language grammar in a formal
way that allows for the enumeration of all variants for any security flaw.
Such a grammar should assume a typical memory model for the languages
where it is to be used (e.g general procedural languages which have a
stack based implementation).  So then we assume:
      \begin{enumerate}
         \item Memory model:
            \begin{enumerate}
               \item data declaration: array $|$ scalar $|$ pointer $|$ literal $|$ function pointer
               \item executable statements: branching control flow $|$ loops $|$ function $|$ executable statements
               \item code: data declaration $|$ executable statements 
               \item function: code, return address
             % \item program: data $|$ function
            \end{enumerate}
      \end{enumerate}
In this memory model a write to the {\em return address} is a security flaw (violation).
The development of a grammar specific to security flaws that uses a specific memory model
may be a way of enumerating all the variations of a specific security flaws instantiation.
Generating code for each enumerated security flaw in the grammar would define all possible
security flaw programs.  Limiting the depth of the recursion would make this a finite
number of programs.

   \item For the specific case of bug seeding, the previous item (above) is not the same
         as requiring an enumeration of all valid  programs over
         a defined grammar, because we start with an existing application which defines
         the set of all vulnerabilities and we define only local modifications that
         are used to generate a new program.  I expect that because of the local nature
         of the transformations to an existing (valid) program, that we should not
         generate invalid programs.  It would be nice to prove this detail (or 
         at least argue it better).

   \item ROSE is well suited to construct these sorts of tools this since it has a high
         level AST representation, most of the program analysis required, the AST subtree
         copy mechanisms, the transformation support to introduce the security flaws, and 
         the code generation support to many different generate different versions of
         the security flaw.

\end{enumerate}
