\documentclass[natbib]{article}
\usepackage{microtype}
\usepackage{lmodern}
\usepackage{url}
\usepackage{xspace}
\usepackage{calc}
\usepackage{enumerate}
\usepackage{listings}
\usepackage{amsmath,amssymb}
\usepackage{rotating}
\usepackage{colortbl}
\usepackage{pifont}
\usepackage{tikz}
%\usetikzlibrary{shapes,shadows,arrows,calc,positioning,fit,matrix,mindmap,trees}
%\usepackage{pgfplots}
%\usepackage{pgfplotstable}
\usepackage{booktabs}
\usepackage{natbib}
\usepackage{colortbl}
% pantone colors

% More sensible defaults akin to \sloppy
% \tolerance 1414
% \hbadness 1414
% \emergencystretch 1.5em
% \hfuzz 0.3pt
% \widowpenalty=10000
% \clubpenalty=10000
% \vfuzz
% \hfuzz
% \raggedbottom

\newcommand{\ignore}[1]{}
\newcommand{\st}{\textit{s.\,t.}\xspace}
\newcommand{\eg}{\textit{e.\,g.}\xspace}
\newcommand{\ie}{\textit{i.\,e.}\xspace}
\newcommand{\cf}{\textit{cf.}\xspace}

\newcommand{\blackarrow}{{\color{black} \Pisymbol{pzd}{217}}}
\newcommand{\redarrow}{{\color{DarkRed} \Pisymbol{pzd}{217}}}
\newcommand{\minibox}[2]{\begin{minipage}{#1}\raggedright #2\end{minipage}}

%\newcommand{\fixme}[1]{\begin{tikzpicture}
%\node[bottom color=red!80!white, top color=red!70!black, rounded corners,
%      font=\bf\color{white}\footnotesize] {
%  \begin{minipage}{.75\columnwidth}
%    FIXME\\
%    #1
%  \end{minipage}
%};
%\end{tikzpicture}
%}

\lstset{
  language=C,
  basicstyle=\small,%\scriptsize, %\footnotesize\ttfamily,
  keywordstyle={\bf},
  keywordstyle={[2]\it},%\color{Blue!40!black}},
  breaklines=true,
  identifierstyle=,
  stringstyle=\bf,
  commentstyle=\it\color{black!80},
  captionpos=b,
  numbers=left,
  stepnumber=3,
  columns=fullflexible
}

\begin{document}
\title{CodeThorn}

\author{\small Markus Schordan, Adrian Prantl, Marc Jasper}
%\end{tabular}
\date{\today}

\maketitle

\begin{abstract}

CodeThorn is a tool for analyzing C/C++ programs, combining approaches
from data flow analysis, constraint-based analysis, and model
checking. The main focus in the development of CodeThorn is the
exploration of program analysis algorithms in combining above
approaches and to explore methods for combining static analsyis with
methods for verification of software.

The input language is currently restricted to a subset of C.

\end{abstract}

%-------------------------------------------------------------------------

\section{Introduction}
\label{sec:intro}

CodeThorn was initially developed as a tool for exploring approaches
for reachbility analysis and verification of linear temporal logic
formulae based for finite state systems. This was later extended to
also perform specialisation of programs and program equivalence
checking. CodeThorn is based on ROSE \cite{roseWWW} and uses the ROSE abstract syntax
tree as basis for its input. A number of components have been moved
from CodeThorn to ROSE over time. What remains are command line
options that allow to access those features conveniently or also to
reproduce some published results.

\subsection{Benchmarks}

We have found the RERS-challenge
problems \footnote{http://www.rers-challenge.org} being an excellent
guidence in crafting this early version of the tool and investigating
the impact and performance of each of the approaches on the overall
results. For the RERS programs linear temporal logic formulae are
provided. This allows to verify behavioral properties of these
programs. Reachability properties can be verified by checking the
reachability of failing assertions.

For program equivalence checking and data race detection the
Polybench/C 3.2 suite has provided a good basis for evaluation. By
generating various polyhedral variants of the 30+ benchmarks CodeThorn
can be used to check the equivalence of two given programs and verify
whether the optimizations are semantics preserving. Furthermore
parallel OpenMP for loops are recognized and can be checked not to
introduce data races. Currently these approaches are extended to
address other large scale applications.

\section{Command Line Options}
\begin{verbatim}
 -h [ --help ]                         produce this help message
  --rose-help                           show help for compiler frontend options
  -v [ --version ]                      display the version
  --internal-checks                     run internal consistency checks 
                                        (without input program)
  --verify arg                          verify all LTL formulae in the file 
                                        [arg]
  --ltl-verifier arg                    specify which ltl-verifier to use 
                                        [=1|2]
  --csv-ltl arg                         output LTL verification results into a 
                                        CSV file [arg]
  --debug-mode arg                      set debug mode [arg]
  --csv-spot-ltl arg                    output SPOT's LTL verification results 
                                        into a CSV file [arg]
  --csv-assert arg                      output assert reachability results into
                                        a CSV file [arg]
  --csv-assert-live arg                 output assert reachability results 
                                        during analysis into a CSV file [arg]
  --csv-stats arg                       output statistics into a CSV file [arg]
  --csv-stats-size-and-ltl arg          output statistics regarding the final 
                                        model size and results for LTL 
                                        properties into a CSV file [arg]
  --csv-stats-cegpra arg                output statistics regarding the 
                                        counterexample-guided prefix refinement
                                        analysis (cegpra) into a CSV file [arg]
  --tg1-estate-address arg              transition graph 1: visualize address 
                                        [=yes|no]
  --tg1-estate-id arg                   transition graph 1: visualize estate-id
                                        [=yes|no]
  --tg1-estate-properties arg           transition graph 1: visualize all 
                                        estate-properties [=yes|no]
  --tg1-estate-predicate arg            transition graph 1: show estate as 
                                        predicate [=yes|no]
  --tg2-estate-address arg              transition graph 2: visualize address 
                                        [=yes|no]
  --tg2-estate-id arg                   transition graph 2: visualize estate-id
                                        [=yes|no]
  --tg2-estate-properties arg           transition graph 2: visualize all 
                                        estate-properties [=yes|no]
  --tg2-estate-predicate arg            transition graph 2: show estate as 
                                        predicate [=yes|no]
  --tg-trace arg                        generate STG computation trace 
                                        [=filename]
  --colors arg                          use colors in output [=yes|no]
  --report-stdout arg                   report stdout estates during analysis 
                                        [=yes|no]
  --report-stderr arg                   report stderr estates during analysis 
                                        [=yes|no]
  --report-failed-assert arg            report failed assert estates during 
                                        analysis [=yes|no]
  --precision-exact-constraints arg     (experimental) use precise constraint 
                                        extraction [=yes|no]
  --tg-ltl-reduced arg                  (experimental) compute LTL-reduced 
                                        transition graph based on a subset of 
                                        computed estates [=yes|no]
  --semantic-fold arg                   compute semantically folded state 
                                        transition graph [=yes|no]
  --semantic-elimination arg            eliminate input-input transitions in 
                                        STG [=yes|no]
  --semantic-explosion arg              semantic explosion of input states 
                                        (requires folding and elimination) 
                                        [=yes|no]
  --post-semantic-fold arg              compute semantically folded state 
                                        transition graph only after the 
                                        complete transition graph has been 
                                        computed. [=yes|no]
  --report-semantic-fold arg            report each folding operation with the 
                                        respective number of estates. [=yes|no]
  --semantic-fold-threshold arg         Set threshold with <arg> for semantic 
                                        fold operation (experimental)
  --post-collapse-stg arg               compute collapsed state transition 
                                        graph after the complete transition 
                                        graph has been computed. [=yes|no]
  --viz arg                             generate visualizations (dot) outputs 
                                        [=yes|no]
  --viz-cegpra-detailed arg             generate visualization (dot) output 
                                        files with prefix <arg> for different 
                                        stages within each loop of cegpra.
  --update-input-var arg                For testing purposes only. Default is 
                                        Yes. [=yes|no]
  --run-rose-tests arg                  Run ROSE AST tests. [=yes|no]
  --reduce-cfg arg                      Reduce CFG nodes which are not relevant
                                        for the analysis. [=yes|no]
  --threads arg                         Run analyzer in parallel using <arg> 
                                        threads (experimental)
  --display-diff arg                    Print statistics every <arg> computed 
                                        estates.
  --solver arg                          Set solver <arg> to use (one of 1,2,3).
  --ltl-verbose arg                     LTL verifier: print log of all 
                                        derivations.
  --ltl-output-dot arg                  LTL visualization: generate dot output.
  --ltl-show-derivation arg             LTL visualization: show derivation in 
                                        dot output.
  --ltl-show-node-detail arg            LTL visualization: show node detail in 
                                        dot output.
  --ltl-collapsed-graph arg             LTL visualization: show collapsed graph
                                        in dot output.
  --input-values arg                    specify a set of input values (e.g. 
                                        "{1,2,3}")
  --input-values-as-constraints arg     represent input var values as 
                                        constraints (otherwise as constants in 
                                        PState)
  --input-sequence arg                  specify a sequence of input values 
                                        (e.g. "[1,2,3]")
  --arith-top arg                       Arithmetic operations +,-,*,/,% always 
                                        evaluate to top [=yes|no]
  --abstract-interpreter arg            Run analyzer in abstract interpreter 
                                        mode. Use [=yes|no]
  --rers-binary arg                     Call rers binary functions in analysis.
                                        Use [=yes|no]
  --print-all-options arg               print all yes/no command line options.
  --eliminate-arrays arg                transform all arrays into single 
                                        variables.
  --annotate-results arg                annotate results in program and output 
                                        program (using ROSE unparser).
  --generate-assertions arg             generate assertions (pre-conditions) in
                                        program and output program (using ROSE 
                                        unparser).
  --rersformat arg                      Set year of rers format (2012, 2013).
  --max-transitions arg                 Passes (possibly) incomplete STG to 
                                        verifier after max transitions 
                                        (default: no limit).
  --max-iterations arg                  Passes (possibly) incomplete STG to 
                                        verifier after max loop iterations 
                                        (default: no limit). Currently requires
                                        --exploration-mode=loop-aware.
  --max-transitions-forced-top arg      same as max-transitions-forced-top1 
                                        (default).
  --max-transitions-forced-top1 arg     Performs approximation after <arg> 
                                        transitions (only exact for 
                                        input,output) (default: no limit).
  --max-transitions-forced-top2 arg     Performs approximation after <arg> 
                                        transitions (only exact for 
                                        input,output,df) (default: no limit).
  --max-transitions-forced-top3 arg     Performs approximation after <arg> 
                                        transitions (only exact for 
                                        input,output,df,ptr-vars) (default: no 
                                        limit).
  --max-transitions-forced-top4 arg     Performs approximation after <arg> 
                                        transitions (exact for all but 
                                        inc-vars) (default: no limit).
  --max-transitions-forced-top5 arg     Performs approximation after <arg> 
                                        transitions (exact for input,output,df 
                                        and vars with 0 to 2 assigned values)) 
                                        (default: no limit).
  --max-iterations-forced-top arg       Performs approximation after <arg> loop
                                        iterations (default: no limit). 
                                        Currently requires --exploration-mode=l
                                        oop-aware.
  --variable-value-threshold arg        sets a threshold for the maximum number
                                        of different values are stored for each
                                        variable.
  --dot-io-stg arg                      output STG with explicit I/O node 
                                        information in dot file [arg]
  --dot-io-stg-forced-top arg           output STG with explicit I/O node 
                                        information in dot file. Groups 
                                        abstract states together. [arg]
  --stderr-like-failed-assert arg       treat output on stderr similar to a 
                                        failed assert [arg] (default:no)
  --rersmode arg                        sets several options such that 
                                        RERS-specifics are utilized and 
                                        observed.
  --rers-numeric arg                    print rers I/O values as raw numeric 
                                        numbers.
  --exploration-mode arg                 set mode in which state space is 
                                        explored ([breadth-first], depth-first,
                                        loop-aware)
  --eliminate-stg-back-edges arg         eliminate STG back-edges (STG becomes 
                                        a tree).
  --spot-stg arg                         generate STG in SPOT-format in file 
                                        [arg]
  --dump-sorted arg                      [experimental] generates sorted array 
                                        updates in file <file>
  --dump-non-sorted arg                  [experimental] generates non-sorted 
                                        array updates in file <file>
  --print-update-infos arg              [experimental] print information about 
                                        array updates on stdout
  --verify-update-sequence-race-conditions arg
                                        [experimental] check race conditions of
                                        update sequence
  --rule-const-subst arg                 [experimental] use const-expr 
                                        substitution rule <arg>
  --limit-to-fragment arg               the argument is used to find fragments 
                                        marked by two prgagmas of that '<name>'
                                        and 'end<name>'
  --rewrite                             rewrite AST applying all rewrite system
                                        rules.
  --normalize arg                       normalize AST before analysis.
  --specialize-fun-name arg             function of name [arg] to be 
                                        specialized
  --specialize-fun-param arg            function parameter number to be 
                                        specialized (starting at 1)
  --specialize-fun-const arg            constant [arg], the param is to be 
                                        specialized to.
  --specialize-fun-varinit arg          variable name of which the 
                                        initialization is to be specialized 
                                        (overrides any initializer expression)
  --specialize-fun-varinit-const arg    constant [arg], the variable 
                                        initialization is to be specialized to.
  --iseq-file arg                       compute input sequence and generate 
                                        file [arg]
  --iseq-length arg                     set length [arg] of input sequence to 
                                        be computed.
  --iseq-random-num arg                 select random search and number of 
                                        paths.
  --error-function arg                  detect a verifier error function with 
                                        name [arg] (terminates verification)
  --inf-paths-only arg                  recursively prune the graph so that no 
                                        leaves exist [=yes|no]
  --std-io-only arg                     bypass and remove all states that are 
                                        not standard I/O [=yes|no]
  --std-in-only arg                     bypass and remove all states that are 
                                        not input-states [=yes|no]
  --std-out-only arg                    bypass and remove all states that are 
                                        not output-states [=yes|no]
  --check-ltl arg                       take a text file of LTL I/O formulae 
                                        [arg] and check whether or not the 
                                        analyzed program satisfies these 
                                        formulae. Formulae should start with 
                                        '('. Use "csv-spot-ltl" option to 
                                        specify an output csv file for the 
                                        results.
  --check-ltl-sol arg                   take a source code file and an LTL 
                                        formulae+solutions file ([arg], see 
                                        RERS downloads for examples). Display 
                                        if the formulae are satisfied and if 
                                        the expected solutions are correct.
  --ltl-in-alphabet arg                 specify an input alphabet used by the 
                                        LTL formulae (e.g. "{1,2,3}")
  --ltl-out-alphabet arg                specify an output alphabet used by the 
                                        LTL formulae (e.g. "{19,20,21,22,23,24,
                                        25,26}")
  --io-reduction arg                    (work in progress) reduce the 
                                        transition system to only 
                                        input/output/worklist states after 
                                        every <arg> computed EStates.
  --keep-error-states arg               Do not reduce error states for the LTL 
                                        analysis. [=yes|no]
  --no-input-input arg                  remove transitions where one input 
                                        states follows another without any 
                                        output in between. Removal occurs 
                                        before the LTL check. [=yes|no]
  --with-counterexamples arg            adds counterexample traces to the 
                                        analysis results. Applies to reachable 
                                        assertions (work in progress) and 
                                        falsified LTL properties. [=yes|no]
  --with-assert-counterexamples arg     report counterexamples leading to 
                                        failing assertion states (work in 
                                        progress) [=yes|no]
  --with-ltl-counterexamples arg        report counterexamples that violate LTL
                                        properties [=yes|no]
  --counterexamples-with-output arg     reported counterexamples for LTL or 
                                        reachability properties also include 
                                        output values [=yes|no]
  --check-ltl-counterexamples arg       report ltl counterexamples if and only 
                                        if they are not spurious [=yes|no]
  --reconstruct-assert-paths arg        takes a result file containing paths to
                                        reachable assertions and tries to 
                                        reproduce them on the analyzed program.
                                        [=file-path]
  --reconstruct-max-length arg          parameter of option "reconstruct-input-
                                        paths". Sets the maximum length of 
                                        cyclic I/O patterns found by the 
                                        analysis. [=pattern_length]
  --reconstruct-max-repetitions arg     parameter of option "reconstruct-input-
                                        paths". Sets the maximum number of 
                                        pattern repetitions that the search is 
                                        following. [=#pattern_repetitions]
  --pattern-search-max-depth arg        parameter of the pattern search mode. 
                                        Sets the maximum input depth that is 
                                        searched for cyclic I/O patterns 
                                        (default: 10).
  --pattern-search-repetitions arg      parameter of the pattern search mode. 
                                        Sets the number of unrolled iterations 
                                        of cyclic I/O patterns (default: 100).
  --pattern-search-max-suffix arg       parameter of the pattern search mode. 
                                        Sets the maximum input depth of the 
                                        suffix that is searched for failing 
                                        assertions after following an 
                                        I/O-pattern (default: 5).
  --pattern-search-asserts arg          reads a .csv-file (one line per 
                                        assertion, e.g. "1,yes"). The pattern 
                                        search terminates early if traces to 
                                        all errors with "yes" entries have been
                                        found. [=file-path]
  --pattern-search-exploration arg      exploration mode for the pattern 
                                        search. Note: all suffixes will always 
                                        be checked using depth-first search. 
                                        [=depth-first|breadth-first]
  --refinement-constraints-demo arg     display constraints that are collected 
                                        in order to later on help a refined 
                                        analysis avoid spurious 
                                        counterexamples. [=yes|no]
  --cegpra-ltl arg                      Select the ID of an LTL property that 
                                        should be checked using cegpra (between
                                        0 and 99).
  --cegpra-ltl-all arg                  Check all specified LTL properties 
                                        using cegpra [=yes|no]
  --cegpra-max-iterations arg           Select a maximum number of 
                                        counterexamples anaylzed by cegpra 
                                        (default: no limit).
  --set-stg-incomplete arg              set to true if the generated STG will 
                                        not contain all possible execution 
                                        paths (e.g. if only a subset of the 
                                        input values is used). [=yes|no]
  --determine-prefix-depth arg          if possible, display a guarantee about 
                                        the length of the discovered prefix of 
                                        possible program traces. [=yes|no]
  --minimize-states arg                 does not store single successor states 
                                        (minimizes number of states).
\end{verbatim}

\section{Analysis Overview}

The analysis is performed in five phases

\begin{enumerate}
\item Syntactic and semantic analysis of the input program (ROSE). The program is analyzed and represented in memory as annotated abstract syntax tree (AST).
\item Control flow analysis. We compute a control flow graph (CFG) for the AST. Transitions, as computed for the state transition graph in the next phase, correspond to edges in the CFG.
\item Computation of the state transition system.
\item LTL-Checking. Input to the LTL-Checking phase is the state transition system and the LTL-formulae.
\item Reporting of analysis results. Assert reachability is computed based on the transition system. Results for LTL-formulae are computed solely by the LTL-checker.
\end{enumerate}

Important properties of the perfomed analysis are

\subsection{Program State}

A program state consists of a label (lab), a property state (pstate),
a constraint set (cset), an IO property (io). $PState = Var
\rightarrow Val$ where $Val$ is either a constant or $top$. $Val$ is a
lifted integer set. An execution state is defined as $EState = Lab \times
PState \times Constraints \times IO$ where $Constraints$ is a set of
constraints, and $IO$ determines whether one of the variables in $PState$ is
an input or output variable. More specifically, whether a variable is
read from stdin, or printed to stdout or stderr. Furthermore it
determines whether the state produces an output which is caused by a
failed assert.

\ignore{
Hence, the analysis information is represented as:

\begin{itemize}
\item Label: num
\item PState: VarId $\rightarrow$ Val
\item SState: Label $\times$ PStateId $\times$ ConstraintId $\times$ IOId
\end{itemize}
}

\subsection{IO Property - Input/Output and Exit Information}

This property represents the information that the operation is either
an input, an output, or a failed-assert operation. From the label and
the associated statement, details as relevant to reporting the
analysis results can be extracted. For example, in the RERS-problems
we find the information for a failed-assert by a) using the label to
find the corresponding statement and b) the associated C error label
is then found in a look-up operation on the ROSE AST.

\subsection{The State Transition System}

\newcommand{\deqop}[0]{\#\#}
\newcommand{\eq}[0]{=}

The State Transition System is represented as a graph where nodes
represent system states and edges represent the transition between two
states as computed by the evaluations of the respective transfer
function. They are therefore annotated with the respective statement
which has been analyzed by the transfer function.

\bibliographystyle{unsrtnat}
\bibliography{codethorn}

\end{document}
