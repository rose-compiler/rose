\documentclass[natbib]{article}
\usepackage{microtype}
\usepackage{lmodern}
\usepackage{url}
\usepackage{xspace}
\usepackage{calc}
\usepackage{enumerate}
\usepackage{listings}
\usepackage{amsmath,amssymb}
\usepackage{rotating}
\usepackage{colortbl}
\usepackage{pifont}
\usepackage{tikz}
%\usetikzlibrary{shapes,shadows,arrows,calc,positioning,fit,matrix,mindmap,trees}
%\usepackage{pgfplots}
%\usepackage{pgfplotstable}
\usepackage{booktabs}
\usepackage{natbib}
\usepackage{colortbl}
% pantone colors

% More sensible defaults akin to \sloppy
% \tolerance 1414
% \hbadness 1414
% \emergencystretch 1.5em
% \hfuzz 0.3pt
% \widowpenalty=10000
% \clubpenalty=10000
% \vfuzz
% \hfuzz
% \raggedbottom


\newcommand{\st}{\textit{s.\,t.}\xspace}
\newcommand{\eg}{\textit{e.\,g.}\xspace}
\newcommand{\ie}{\textit{i.\,e.}\xspace}
\newcommand{\cf}{\textit{cf.}\xspace}

\newcommand{\blackarrow}{{\color{black} \Pisymbol{pzd}{217}}}
\newcommand{\redarrow}{{\color{DarkRed} \Pisymbol{pzd}{217}}}
\newcommand{\minibox}[2]{\begin{minipage}{#1}\raggedright #2\end{minipage}}

\lstset{
  language=C,
  basicstyle=\small,%\scriptsize, %\footnotesize\ttfamily,
  keywordstyle={\bf},
  keywordstyle={[2]\it},%\color{Blue!40!black}},
  breaklines=true,
  identifierstyle=,
  stringstyle=\bf,
  commentstyle=\it\color{black!80},
  captionpos=b,
  numbers=left,
  stepnumber=3,
  columns=fullflexible
}

\begin{document}
\title{The CodeThorn Program Analyzer}

\author{
\small
\begin{tabular}{ll}
Markus Schordan                        & Adrian Prantl\\
Institute of Computer Science          & Center for Applied Scientific Computing\\
UAS Technikum Wien                     & Lawrence Livermore National Laboratory\\
\texttt{schordan@technikum-wien.at}   & \texttt{adrian@llnl.gov}\\
\end{tabular}
}
\date{\today}

\maketitle

\begin{abstract}
  CodeThorn.
\end{abstract}

%-------------------------------------------------------------------------

\newcommand{\fixme}[1]{\begin{tikzpicture}
\node[bottom color=red!80!white, top color=red!70!black, rounded corners,
      font=\bf\color{white}\footnotesize] {
  \begin{minipage}{.75\columnwidth}
    FIXME\\
    #1
  \end{minipage}
};
\end{tikzpicture}
}

\section{Introduction}
\label{sec:intro}

CodeThorn is a tool for analyzing C/C++ programs, combining approaches
from data flow analysis, constraint-based analysis, and model
checking. The main focus in the development of CodeThorn is the
exploration of program analysis algorithms in combining above
approaches for its application in analyzing and optimizing
high-performance applications (HPC) and embedded systems
applications. When analyzing HPC applications one important aspect is
the analysis of parallelism in programs. For embedded systems timing
behaviour, e.g. wort-case execution time, is of equal
importance. Since processors, as they are used in embedded systems,
are becoming increasingly parallel, parallelism at the application
level is also becoming increasingly important. However, many HPC
applications are massively parallel and as such have no complicated
control structure, in difference to embedded systems control software
which usually shows complex control patterns which would not be
suitable for executing on massively parallel hardware (such as modern
GPUs).

CodeThorn has grown out of both those application areas, from
analyzing and optimizing source-to-source HPC applications \cite{MS-QSMK04CPA} with
ROSE-based tools \cite{ROSEWEBSITE,MS-SCAM05} and analysing embedded systems
software with SATIrE (which integrates ROSE and PAG \cite{PAG}) and
which has been applied in performing loop bound analysis (TuBound \cite{Prantl:SK08}, \cite{Kirner:SOSYM2010}) timiing
analysis (e.g. WCET challenge 2008/9 \cite{Prantl:WCET2009}).

Since verification of software is becoming increasingly important in
all areas where software analysis can be applied to aid the proper
development of software, we have recently begun the development of
CodeThorn to investige approaches that allow to combine approaches
from different areas and which may become suitable in different areas
of HPC and embedded systems.

We currently restrict the input language to a small subset of C as we
are also using it when normalizing compute intensive code for
optimizations. The RERS 2012 subset is therefore well suited for our
initial version. Future work will include porting our analyses from
SATIrE to CodeThorn (e.g. shape analysis \cite{shapeanalysis},
points-to analysis \cite{pointstoanalysis}, interval analysis \cite{intervalanalysis}).

In CodeThorn (in its present version 1.1) we integrate three approaches:

\begin{itemize}
\item Data-flow analysis (inter-procedural flow-senstive analysis, computing property states) 
\item Constraint-based analysis (extracting constraints from a given program and using those in narrowing down possible values of input variables on different branches)
\item Model checking (verifying LTLs using a state transition graph)
\end{itemize}

In the following sections we will describe briefly each approach and
how we combine them in analyzing the RERS challenge-problems. We have
found the RERS-challenge problems being an excellent guidence in
crafting this early version of the tool and investigating the impact
and performance of each of the approaches on the overall results, and
in particular the opportunities we see emerge and limitations we
currently face in respect to verfication. We believe that insights
gained in the work on the RERS challenge-problems will transfer to
analyses as we shall use them in above mentioned areas of HPC and
embedded systems.

\section{The System State Transition Graph}

\newcommand{\deqop}[0]{\#\#}
\newcommand{\eq}[0]{=}

The Transition Graph is a graph where nodes represent system states
and edges represent the transition between two states as computed by
the evaluations of the respective transfer function. They are
therefore annotated with the respective statement which has been
analyzed by a transfer function.

The nodes represent System States and consist of:
\begin{itemize}
  \item a label (representing an abstract program counter)
  \item a Property State
  \item a Constraint Set
  \item Input/Output and Exit information of the program.
 \end{itemize}

\subsection{Folded System State Transition Graph}

\begin{figure*}[t]
\centering
\includegraphics[width=0.7\textwidth]{gfx/basictest15_transitiongraph2.pdf}
\caption{Folded System State Transition Graph}
\end{figure*}

\begin{figure*}[t]
\centering
\includegraphics[width=0.7\textwidth]{gfx/basictest10f_transitiongraph2.pdf}
\caption{Folded System State Transition Graph (needs to reduced in size)}
\end{figure*}

\clearpage
\subsection{Property States}

\fixme{describe property states, const-int-lattice}

\subsection{Minimal Constraint Sets}
A minimal constraint set (MCS) is a minimal representation of
information about the values of variables and relations beetween the
variables with a minimal number of operators. The minimal number is
important for using a minimal amount of memory and for a proper
sorting criteria to be used in a sorted data structure.

The operators can range on variables and constants. Let x,y,z denote variables, and c, d constants with c!=d.
The constraints of interest are : 
\begin{itemize}
\item $x\eq c$: const-equality constraint
\item $x\neq c$, const-inequality constraint
\item $x\eq y$, var-equality constraint
\item $x\neq y$, var-inequality constraint
\end{itemize}

Additionally a set can be marked to represent an disequality, meaning that an equality and an inquality on the same variable have been added to the set. This is represented as \{\#\#\}.

Set union of MCS1 and MCS2 means inserting one constraint after another of MCS2 into MCS1.

The empty constraint set is minimal.

We define the consistency of a MCS by definining the effect of adding a new constraint to a set and define all cases.
Secondly, as we shall see, to maintain consistency, we also need to define effects for removing a constraint.

\fixme{rewrite to be become more concise}

1) Adding constraints to a MCS:

Adding a constraint C to an empty MCS gives a MCS.

\[\{\}+\{x=c\}\rhd \{x=c\}\]
\[\{\}+\{x\neq c\}\rhd \{x\neq c\}\]
\[\{x=c\}+\{x=c\}\rhd \{x=c\}\]
\[\{x=c\}+\{x=d\}\rhd \{\deqop\} with c<d\]
\[\{x=c\}+\{x\neq c\}\rhd \{\deqop\}\]
\[\{x\neq c\}+\{x=c\}\rhd \{\deqop\}\]
\[\{x\neq c\}+\{x\neq d\}\rhd \{x\neq c,x\neq d\}\]
No operations are defined on a MCS which includes a \{\deqop\} constraint.
\[\{x\neq c\}+\{x=d\}\rhd \{x=d\}\] (we drop the inequality on x)
\[\{x\neq c\}+\{x=d\}\rhd \{x\neq c,x=d\}\]
Lemma1: for each variable there can exist at most one const-equality constraint $x=c$.
Lemma2: if there exists a const-equality constraint for variable x, no const-inequality constraints can exist for this variable (and any other variable which is equal to that variable, i.e. an y=x constraint exists).
\[\{\}+\{x=y\}\rhd \{x=y\} if lex(x)<lex(y)\]
\[\{\}+\{x=y\}\rhd \{y=x\} if lex(x)\geq lex(y)\]

the rhs argument of '+' is sorted such that:
\[x=y \rhd  x=y if lex(x)<=lex(y)\]
\[x=y \rhd  y=x if lex(x)>lex(y)\]

\[\{x=y\}+\{x=c\}\rhd \{x=y,x=c\}\]
\[\{x=y\}+\{x\neq c\}\rhd \{x=y,x\neq c\}\]
\[\{x=y\}+\{y=c\}\rhd \{x=y,x=c\}\]
\[\{x=y\}+\{y\neq c\}\rhd \{x=y,x\neq c\}\]

\[add(MCS,\{x=y\})\rhd \{\deqop\} if constants(x,MCS)\neq constants(y,MCS)\]
\[add(MCS,\{x=y\})\rhd MCS+\{x=y\} if constants(x,MCS)=constants(y,MCS)\]

2) Removing constraints from an MCS:
\[remove(MCS,\{x\neq c\})\rhd MCS-\{x\neq c\}\]
\[remove(MCS,\{x=c\})\rhd MCS-\{x=c\}\]
remove($MCS,\{x=y\}$)$\rhd$  (lazy reorganization of constraint set)\\
  if(constraintexists($x,MCS\backslash\{x=y\}$) and constraintexists($y,MCS\backslash\{x=y\}$)\\
     $\rhd$  moveconstconstraints($x,y,MCS)\backslash\{x=y\}$)\\
  if(constraintexists($x,MCS\backslash\{x=y\}$) and not constraintexists($y,MCS\backslash\{x=y\}$)\\
     $\rhd$  moveconstconstraints($y,x,MCS)\backslash\{x=y\}$)\\
  if(not constraintexists($x,MCS\backslash\{x=y\}$) and not constraintexists($y,MCS\backslash\{x=y\}$)\\
     $\rhd$  $MCS\backslash\{x=y\}$\\
remove($MCS,\{x\neq y\}$) $\rhd$ $MCS-\{x\neq y\}$\\

We move all const-constraints of one variable to the other in case x=y was the last constraint concerning this variable. If no constraint remains for both variables (case 3), we can savely remove the x=y constraint.

We could achieve also a canonical representation if the moving of constraints would be performed when adding constraints (note that we order $x=y$ constraints according to the lexical order of x,y).

Auxiliary functions:
lex(x) : we use the ids as order criteria 
constraintexists(x,MCS) : constraint with this variable exists
constants(x,MCS) : number of constants that can be computed for this variable

\fixme{move to more prominent location}

Data Representation:
PState = Var x Val where Val is either a constant or top. Val is a lifted integer set.
SState = Lab x PState x Constraints x IO

Constraints is a set of constraints (see above).

IO determines whether one of the variables in PState is an input or output variable. More specifically, whether a variable is read from stdin, or printed to stdout or stderr. Furthermore it determines whether the state produces an output which is caused by a failed assert.


Labels are unique and represented as numbers. Ordering is therefore the same as for numbers.

IO is ordered by the variable it is concerned of and by the internal number that is associated with the IO-operation.

Each constraints set is associated with an id. The id is used in PState.

Hence, the analysis information is represented as:

\begin{itemize}
\item Label: num
\item PState: set of VarId x ValId
\item SState: set of Label x PStateId x ConstraintId x IOId
\end{itemize}

\subsection{Input/Output and Exit Information}

\section{Solver}

\fixme{describe algorithm}

\section{Implementation}

\fixme{describe data organization, approach for reducing redundancy, etc.}

A PState is sorted by the lexical order of the variables.
PState$\eq$ PState: two PStates are equal if all variable bindings are equal.
PState$<$ one PState is smaller than another if, considering the sequence of variables as a string, the string is smaller than the other according to lexical order.

% !TEX root = codethorn.tex
\section{Verifying LTL formul\ae}

\newcommand{\ffalse}{\ensuremath{\mathit{false}}}
\newcommand{\ttrue}{\ensuremath{\mathit{true}}}

At its heart CodeThorn uses a dataflow-based approach to verify LTL
formul\ae\ on the State Transition Graph. The algorithm consists of
two layers. At the outer layer, the abstract syntax tree of the
formula is traversed in bottom-up order. Every sub-expression of the
LTL formula found by the traversal is reduced to a single value in the
Boolean lattice (\cf Fig.~\ref{fig:bool_lattice}) for each state in
the State Transition Graph. This evaluation is performed using a
dataflow analysis, which represents the inner layer of the algorithm.

\begin{figure}
  \centering
   \begin{tikzpicture}[scale=.9]
    \small
    \node (top) {$\top$} 
    child {node (-1)    {\ffalse}}
    child {node (0)     {}
      edge from parent[draw=none]
      child {node (bot) {$\bot$} 
      edge from parent[draw=none]
      }
    }
    child {node (1)    {\ttrue}} 
    ; 
    \draw (-1)    -- (bot);
    \draw (1)     -- (bot);
  \end{tikzpicture} 
  \caption{Boolean lattice the LTL formul\ae\ are reduced to}
  \label{fig:bool_lattice}
\end{figure}

\newcommand{\lub}{\ensuremath{\sqcup}\xspace}
\newcommand{\Lub}{\ensuremath{\bigsqcup}\xspace}
\newcommand{\state}{\ensuremath{\mathit{s}}\xspace}
\newcommand{\STG}{\ensuremath{\mathrm{STG}}\xspace}
\newcommand{\States}{\ensuremath{\mathit{States}}\xspace}
\newcommand{\prop}[1]{\ensuremath{p_{\state,#1}}\xspace} 
\newcommand{\propp}[1]{\ensuremath{p_{\state',#1}}\xspace} 
\newcommand{\G}{\ensuremath{\mathrm{G}}\ }
\newcommand{\F}{\ensuremath{\mathrm{F}}\ }
\newcommand{\X}{\ensuremath{\mathrm{X}}\ }
\newcommand{\R}{\ensuremath{\mathrm{R}}\ }
\newcommand{\U}{\ensuremath{\mathrm{U}}\ }
\newcommand{\WU}{\ensuremath{\mathrm{WU}\ }}

\subsection{A minimum-fixpoint approximation for all-quantified LTL expressions}
The approach we are using is extremely fast, but it computes only an
approximation (a safe upper bound in the Boolean lattice) of the
precise verification result. The approximation happens whenever
multiple paths join at a state, in which case the least upper bound
\lub\ of the results from the individual paths is propagated.

The remainder of this section will discuss how the transfer functions
for each element of the LTL grammar are defined. We are using the
following notational conventions. The
$\STG={\mathit{States},\mathit{Transitions},\mathit{Start}}$ is the
(reduced) State Transition Graph (or Kripke Model
\citep[pg. 27ff]{Clarke1999}). For each state \state\ we have an array of $n$
properties \prop{i}, $i \in [0\dots n]$, where $n+1$ is the number of
sub-expressions in the LTL formula.

\subsubsection{Boolean Operators: \texttt{!}, \texttt{\&}, \texttt{|}}
The Boolean operators over LTL expressions have the least
computationally complex transfer functions as their effect is only
local to each state.
\begin{align*}
e &= !a  & \forall\state\in\States\colon\prop{e} &:= \neg\prop{a}\\
e &= a \& b & \forall\state\in\States\colon\prop{e} &:= \prop{a} \cap \prop{b}\\
e &= a | b & \forall\state\in\States\colon\prop{e} &:= \prop{a} \cup \prop{b}\\
\end{align*}

\subsubsection{The G operator}
The global operator $\G a$ yields true iff the sub-expression $a$ is
true at all states. The transfer function is defined as follows:

\[ e = \G a\colon\qquad\forall\state\in\States\colon\prop{e}
:=\Lub_{\state\prime\in succ(\state)}{\propp{a}} \]

Since we only want to handle all-quantified LTL expressions, it is
safe to join the information from multiple paths using the \lub
operator: If all successors of a node share the yield the same result,
by induction, all paths starting at \state have identical results at
every node, thus we can merge their heads at \state. If the results of
the successors diverge, \state will go to $\top$, (the safe
approximation) and the result at \state is unknown.

Due to the way the logical operations on Boolean lattice are defined,
not all is lost for these states. For example, $\top \vee \ttrue
\equiv \ttrue$ and $\top \wedge \ffalse \equiv \ffalse$, so it is
possible for these unknown states to still be overruled by other
states for which we do have a precise result.

\subsubsection{The X operator}
The next operator $X a$ yields true iff $a$ is true in the next state.
\[ e = \X a \qquad\forall\state\in\States\colon\prop{e}:=\bigcap_{\state\prime\in succ(\state)}{\propp{a}} \]

\subsubsection{The operators F, R, U, and WU}
These operators are computationally more intensive, since we need to
perform a fix point search over the entire state transition graph. Backward.
\begin{tabular}{rlllll}
\toprule
operator & $\mathit{init}$ & $\mathit{start}$ & $\mathit{join}$ &
$\mathit{calc}$ & $\mathit{otherwise}$ \\\midrule

$e = \F a$  & $\ffalse$ & $(\prop{a}==true)$ & $\cup$ & 
$\prop{a}\cup\bigcup_{\state\prime\in succ(\state)}{\prop{e}}$ &\\

$e = a \R b$ & $\bot$ & $(\prop{b}==true)$ & $\cap$ &
$\prop{b} \cap \prop{a} \cup \bigcap_{\state\prime\in succ(\state)}{\prop{e}}$\\

$e = a \U b$ & $\ffalse$ & $(\neg\prop{b}==bot)$ &
$\cup$ & $\prop{a} \cup\bigcap_{\state\prime\in succ(\state)}{\prop{e}}$ & $\ffalse$\\

$e = a \WU b$ & $\bot$ & $(\prop{a}==true)$ & $\cup$ &
$\prop{a} \cup \bigcup_{\state\prime\in succ(\state)}{\prop{e}}$\\\bottomrule
\end{tabular}


\subsubsection{Handling I/O nodes}
\begin{itemize}
\item are propagated to the next I/O state of the same type
\end{itemize}

\subsection{Reducing the State Transition Graph}

\subsection{Debugging: Finding counterexamples}

\bibliographystyle{unsrtnat}
\bibliography{codethorn}

\end{document}
