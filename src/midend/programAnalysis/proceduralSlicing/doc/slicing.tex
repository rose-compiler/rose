\documentclass[11pt,a4paper,twoside]{article}
                                                                                
\usepackage[latin1]{inputenc}
\usepackage[T1]{fontenc}
\usepackage[english]{babel}
 
\usepackage{graphicx}     % If Postscript figures are to be included.
\usepackage{verbatim} 


\begin{document}
\tableofcontents
\newpage
\section{Introduction}
The first section gives a brief introduction to some theory about program slicing. This section is meant to serve as a short background. The next section illustrates this theory through a small example of slicing. The third section describes the slicing class \texttt{Slicing} using ROSE. The different slicing interfaces are described as well as the slicing process itself. Toward the end an example of program slicing is given. \marginpar{\scriptsize Update this if necessary...\normalsize }

% - - - - - - - - - - - - SOME THEORY - - - - - - - - - - - - - - - - - - -
\section{Program Slicing}
The following definitions are taken from \textit{The Compiler Design Handbook: Optimizations and Machine Code Generation}, \cite{chp8}.

Program slicing is extracting statements of a program which are relevant for a given computation. A program can be sliced with respect to a slicing criterion. This slicing criterion can be a pair $<p, V>$, where $p$ is a program point of interest and $V$ is a subset of the program variables. A slice of a program $P$ with respect to the slicing criterion $<p,V>$ is defined as the set of all statements of $P$ that might affect the values of the variables in $V$ used or defined at the program point $p$. More specifically, this is called static backward slicing. Static as it is independent of the input values to the program. Backward because the control flow of the program is considered in reverse while constructing the slice.

\subsection{Static versus Dynamic Slicing}
As mentioned, static slicing is done independently of the input values. That is, the slice will be good for all possible input values. A static slice contains all the statements that may affect the value of a variable at a program point for every possible input, hence a static slice may contain some statements that might not be executed during an actual run of the program.

Dynamic slicing makes use of the information about a particular execution of a program. A dynamic slice with respect to a slicing criterion $<p,V>$, for a particular execution, contains only those statements that actually affect the values of the variables in $V$ at the program point $p$. Dynamic slices are usually smaller than the static.

\subsection{Backward versus Forward Slicing}
A backward slice contains all parts of the program that might directly or indirectly affect the variables at the statement under consideration. A backward slice answers the question: Which statements affect the slicing criterion?

A forward slice with respect to slicing criterion $<p,V>$ contains all parts of the program that might be affected by the variables in $V$ used or defined at the program point $p$. A forward slice answers the question: Which statements will be affected by a slicing criterion?

\subsection{Applications of Program Slicing}
Applications of program slicing include debugging, software maintenance and testing of code. For instance in the process of debugging, programmers unconsciously use a mental form of slicing. That is the programmer ignores some statements and considers a subset of the statements in the code while attempting to localize a bug.


\subsection{Interprocedural Slicing versus Slicing of Object-Oriented Programs}

Intraprocedural slicing, that is slicing within a function, uses control flow and data flow information as intermediate program representation.

Interprocedural slicing, that is slicing across functions and function calls, is complicated by possible side-effects of function calls on variables in interest.

In this context we assume function calls do not have such complicated side-effect and the slicing is interprocedural in the sense that it removes functions not needed for the slice, while keeping unmodified the functions that are called.

To slice object-oriented programs the intermediate representations for the programs need to model classes, inheritance, scoping, persistence, polymorphism and dynamic binding in an effective way.

For more about interprocedural slicing and object-oriented slicing, the reader is referred to other sources, for instance \cite{chp8} and \cite{OO}.


\section{Introductory Example of Static Backward Slicing}
Slicing is easily understood by considering an example. We therefore consider the following code:
\scriptsize
\begin{verbatim}
int main(){

  int y = 6;
  int x = 3;
  int z;
  z = x + y;
  y = 4;
  x = 5;
  x = x + y;
  z = z + 2*x;
  x = x + 5;
  y = x - 3;
  
  return 0;
}
\end{verbatim}
\normalsize
We want to do slicing with respect to the statement \texttt{x = x + 5}. This will be the slicing criterion. The slice we get with respect to this statement is:
\scriptsize
\begin{verbatim}
int main(){

  int y = 6;
  int x = 3;


  y = 4;
  x = 5;
  x = x + y;

  x = x + 5;

  
  return 0;
}
\end{verbatim}
\normalsize
In this resulting code we see that all references to the variable \texttt{z} are not in the slice, as they do not affect the variable \texttt{x} in any way. The variable \texttt{y} is used in the computation of the variable \texttt{x}, which is further used in the slicing criterion statement. Thus this \texttt{y} has to be in the slice.


% - - - - - - - - - - - - - -  Simple Slicing - - - - - - - - - - - -
\section{A Simple Slicing Class}

This slicing class \texttt{Slicing} using ROSE performs static backward slicing on procedural-oriented programs.

The slicing is interprocedural in the sense that it removes functions not needed for the slice, while keeping unmodified the functions that are called.

\subsection{Slicing interfaces/methods}
\label{sec:interface}
There are two interfaces/methods\marginpar{\scriptsize which word? \normalsize} for performing slicing. The function for invoking a complete slice of a \texttt{SgProject} object is
\begin{verbatim}
void Slicing::completeSlice(SgProject* sgproject);
\end{verbatim} 
This function generates a compilable(?)\marginpar{\scriptsize what's the correct\\word?} output file of the sliced program. The file has the same name as the original file, of which the \texttt{SgProject} was created, however, with an added prefix \textit{rose\_} to the name.

The other interface is the function for obtaining the statements that \textit{directly} affect the slicing criterion:
\begin{verbatim}
void Slicing::sliceOnlyStmts(SgProject* sgproject,
                             set<SgNode*>&  stmt_in_slice);
\end{verbatim}
This function does not necessarily include possible control statements which are not directly associated with the slicing criterion, but that are still needed for a complete slice. These statements are returned as a set of \texttt{SgNode} objects in the parameter \texttt{stmt\_\-in\_\-slice}.\\

\noindent
The complete slice is useful for testing the slicing process itself. By both compiling and running the output code, we can see if it produces the same output of the values in the slicing criterion as the original code.

The partial slice, that is the list of statements we get from \texttt{slice\-Only\-Stmts}, is the same as the statements we get from the definition-use association connected to the slicing criterion. Thus this set of statement does not necessarily include control statements unless the control statements are directly related to the input statement. This slicing interface may be useful for debugging: A more sophisticated slicing tool would highlight the statements obtained from this slice in a code editor. Thus the programmer would be able to consider the slice of directly affecting statements for debugging.
 
\subsection{Input file}
The input file, from which we create a \texttt{SgProject} object, has to contain a slicing criterion. The slicing criterion may be one or more statements in the program code, identified as the statements between two pragma declaration statements:
\scriptsize
\begin{verbatim}
   ... 
   y=3;
#pragma start_slicing_criterion
   x=y+5;
#pragma end_slicing_criterion
   x++;
   ...
\end{verbatim}
\normalsize
Both pragma declarations should be declared within the same scope.

If we want to do slicing on certain variables, not on whole statements, this might be achieved by inserting a variable reference expression into the code as a statement and do slicing on this statement:
\scriptsize
\begin{verbatim}
   ...
   y=3;
   x=y+5;
#pragma start_slicing_criterion
   x;
#pragma end_slicing_criterion
   x++;
   ...
\end{verbatim}
\normalsize

\subsection{The Process of Slicing}
The process of slicing which is done by the class \texttt{Slicing} consists of the following steps:
\begin{itemize}
\item[1.] Identify the slicing criterion.
\item[2.] Find the function within which the slicing criterion is.
\item[3.] Find the necessary statements for the slice. Necessary means here all statements that affect the slicing criterion and are thus necessary for the slice.
\item[4.] Mark the nodes in the AST which correspond to the statements found in the above step.

In this traversal we also collect new statements which are not directly related to the slicing criterion, but they appear in control statements and also need to be included as well as the statements affecting these control statements.

Secondly, in this marking traversal we also collect all function reference expressions that will appear in the slice such that we later can identify which functions are called and thus need to be kept and which functions can be removed.
Initially all function definitions, except the function in which the slicing criterion is, are kept in their complete form. Later unnecessary functions are removed. No attempt is done to slice called functions.

\item[5.] With the new statements from the above traversal we enter a while loop: While we still have some statements which need to be included, we find the necessary statements (step 3) and we mark the nodes in the AST corresponding to these statements (step 4).
\item[6.] With the list of function references to functions that are called and are needed for the slice, we thus query all function definitions for additional function reference expressions. This is done in a while loop to handle ``nested'' function calls.

Finally, if the function definitions do not correspond to any functions in the list we end up with, we remove from this node the flag that tells us to keep the node, set in step 4. (This will remove the whole subtree constituting the function definition in the next removing traversal.)
\item[7.] Remove all AST nodes which are not marked to be kept.
\item[8.] Unparse the AST to get the output code.
\end{itemize}

\subsubsection{Identify the slicing criterion}
To find the slicing criterion, we parse the AST in the class \texttt{SlicingCriterion} (subclass of \texttt{AstSimpleProcessing}) and find the statements between the two pragma declarations. These statements are put in a set. This set of statements is returned by the function \texttt{get\_stmts\_between\_pragma()}.

\subsubsection{Finding the function declaration}
To find the function declaration, we use the first statement of the slicing criterion (here \texttt{SgNode* node}) and ask for its function definition through the functions:
\scriptsize
\begin{verbatim}
SgFunctionDeclaration* func_decl = 
                           TransformationSupport::getFunctionDeclaration(node);
SgFunctionDefinition* func_defn = func_decl->get_definition(); 
\end{verbatim}
\normalsize
Using the first statement of the slicing criterion to find the function definition should be sufficient, since all statements in the slicing criterion are in the same basic block, hence in the same function.



\subsubsection{Finding the set of statements from the Definition Use Chain}
With the input of the function definition AST node, we build the Definition Use Chain graph for this function. For each statement in the slicing criterion, we find by using the Definition Use Chain, all statements affecting it. We thus get the set (\texttt{slice} in the class \texttt{FindStatements}) of all statements that affect the slicing criterion statements.

However, no control statements and expressions associated with control statements, which are not \textit{directly} related to the slicing criterion statement are included.
Later in the marking traversal we therefore have to include control statements if a statement in its basic block is in the slice. Furthermore, all statements affecting the control statements have to be included. The above process of finding necessary statements will be repeated for these statements.\\

If the slicing criterion includes a variable, which at an earlier point in the code has been redefined by being the left hand operand of an assign statement and this statement is not the declaration statement, the original (C$++$) declaration of the variable will not be included in the list of statements from the Definition Use Chain. The declaration of a variable is necessary for the output code to compile. When traversing the AST later, we therefore check that all used variables are properly defined.

\subsubsection{Marking Nodes}
When having all statements necessary for a slice with respect to the slicing criterion, we traverse the AST (in the class \texttt{MarkingNodes}) with an \texttt{AstTop\-Down\-Bottom\-Up\-Processing} traversal and compare the AST nodes to this set of statements obtained from the Definition Use Chain.

When we find a node corresponding to a node in the set, we add an attribute (\texttt{KeepAttribute} subclass of \texttt{AstAttribute}) to this node, telling that this is a node we want to keep. In addition, we pass this information down to the subtree starting at this node, so that these nodes also get the attribute which tells us to keep the node. However this might only be done if the node is is not a \texttt{SgBasicBlock}. From basic blocks we cannot unconditionally inherit such information, as basic blocks might contain statements that are not needed in the slice.

Secondly, when an AST node has the attribute ``keep'', the information that we want to keep a node is passed on up in the tree, with a synthesized attribute, making sure that the ``line'' of nodes from this node up to the root is marked with the attribute telling it is a node we want to keep.\\

In this traversal we also take some special considerations as follows in the sections below.\\

\noindent
\textbf{Including control statements}\\
When we see a control statement in the BottomUp traversal and if this node is marked to be kept, we extract the conditional statements within this control statement (conditional expression, test expression, increment expression, etc.). All such statements and expressions are put in a new list. In the same way we found the statements affecting the slicing criterion statements, we find the statements affecting these statements. They are put in a set, and with this set, we once again traverse the AST and mark the nodes which correspond to the statements in the set. This way we include statements affecting the control statements.\\

\noindent
\textbf{Avoiding inclusion of unnecessary statements in a \texttt{SgBasicBlock}}\\
Even though we mark a \texttt{SgBasicBlock} node to be kept, we cannot unconditionally pass on an inherited variable that tells all its children to be kept. This is because such a basic block might include more statements than what is necessary for the slice. We therefore always pass on the inherited value from a \texttt{SgBasicBlock}, telling not to keep the children. Then for each child we compare the nodes to the statement list, and use this criterion (and not the inherited value) to find out whether or not this node should be kept.

However, if we discover a synthesized attribute from one of the children to a basic block telling us to keep the basic block, this information is passed on up in the AST, such that we track the line back to the root.\\

\noindent
\textbf{Interprocedural slicing}\\
The input file which we slice, may contain more than the function we actually do slicing on. As we in the first \texttt{TopDown} traversal do not know if all of the function declarations will be necessary for the slice, we initially assume all of them will be needed and thus mark all function declarations (except the function we do slicing on) to be kept with all of the statements.

If a function declaration is not needed in the output code and this AST node is removed, also the parameter list and function definition will be removed.\\

Since we later want to remove functions that are not needed for the slice, we collect in a list all the functions that are references in the slice. After the traversal we are therefore able to go through this list of functions, and for each of them check if they call additional functions. In case they do, we add these to the list and also check the new functions. This way we find all ``nested'' function calls.

Finally, we query the whole program for function definitions. For those function definitions which are not in the list of functions to keep, we remove the AstAttribute, thus removing the sign of keeping the node. This causes the whole subtree of the function definition to be removed in the last removing traversal.

By comparing the function definitions by name we ensure that we also keep necessary prototype function definitions.\\

\noindent
\textbf{Return statement}\\
Return statements are always returned, no matter if they are needed for the slicing criterion or not. This is because otherwise the function which contains the slicing criterion would not be able to compile properly.\\

\noindent
\textbf{Break and Continue statements}\\
All break and continue statements before the slicing criterion, that is before the pragma declarations, are kept.

\subsubsection{Removal of Nodes}
We traverse the AST a final time to remove all nodes which do not contain the attribute which tells us to keep the node. This traversal is of the type \texttt{AstSimpleProcessing}.

\subsection{Limitations}
We are slicing only procedural-oriented programs. Methods within classes in an object-oriented program are not handled in this slicing class. The reason for this limitation is that the slicing is based on the Definition Use Associations, and slicing object-oriented programs would have required additional representations of the program.

As mentioned earlier, interprocedural slicing means in our context removing functions not needed for the slice, while keeping unmodified functions that are called. Only the function in which the slicing criterion is, is actually sliced, that is unnecessary statements are removed. We are not slicing the called functions due to the complexity of analyzing what would be needed or not needed.

We are neither slicing concurrent nor distributed programs.\\

\noindent
Other limitations concern arrays and pointers. The current aliasing analysis algorithm in ROSE is very conservative, causing pointers to be not precisely handled. Thus slicing arrays and pointers will include more statements than necessary and in effect it will not be any slice.


%------------------------------------------
%             Example of Program Slicing
%------------------------------------------

\section{Example of Program Slicing}
Below is a ROSE program invoking the slicing in the two different ways described in section \ref{sec:interface}: The complete slicing generating a sliced output file, while the other partial slice returns only directly affecting statements. 
\scriptsize
\verbatiminput{main.cpp}
\normalsize

In this example we run the above code with the input file listed below.\footnote{Any input file to this program must end in .c or .C since we make a copy by calling \texttt{frontend} twice. That does not work when the input file ends with for instance .cpp.} As the \texttt{\#pragma} declarations show, the slicing criterion is the statement \texttt{y = x + b} in addition to the output statement \texttt{printf("\%d \%d \%d", y, x, b)}.

This last statement is included to be able to show the output of the values in the slicing criterion. If we compile and run this input file, the output we get is: 54 36 18.
\scriptsize
\verbatiminput{example.c}
\normalsize

Below is the resulting sliced file which is produced from the call \texttt{Slic\-ing::com\-plete\-Slice}. If we compile and run this code, the output from the output statement is: 54 36 18.

This output is identical to the output from the original code, and thus we see that the slicing has preserved the state of the values in that point.
\scriptsize
\verbatiminput{rose_example2.c}
\normalsize
The main points to notice here, are that the variable \texttt{z} and the \texttt{for}-loop defining it, are not included as they have no effect on the slicing criterion. Neither is the function \texttt{inc} included, as it is not called by the sliced code.\\

\noindent
The output to screen from the call \texttt{Slicing::slice\-Only\-Stmts}, that is the list of statements directly affecting the slicing criterion, is 
\scriptsize
\begin{verbatim}
Statements directly affecting the slicing criterion:
int x;
int y =(5);
int b =(7);
x = 4;
b = 2;
int i =(add(x,y));
b = b + x * x;
x = x * i;
y = x + b;
printf(("%d %d %d"),y,x,b);
(No rose_filename written. Only the above list is the output.)
\end{verbatim}
\normalsize
In this list it is worth noticing that the \texttt{if}-statement is not listed. This is because the statement is not directly affecting the slicing criterion: In other words neither of the variables occurring in the slicing criterion used for defining \texttt{y}, that is \texttt{x} and \texttt{b}, does occur in the \texttt{if}-statement. The variable \texttt{y} occurs in the \texttt{if}-statement, however, it is redefined in the slicing criterion, thus the ``old'' value which  is in the \texttt{y} at the program point of the \texttt{if}-statement does not affect this re-definition.

\begin{thebibliography}{}
\bibitem{chp8} \textit{The Compiler Design Handbook: Optimizations and Machine Code Generation}. Edited by Y.N. Srikant, P. Shankar. CRC Press, 2003.
\bibitem{OO} \textit{Slicing Object-Oriented Programs} Jiun-Liang Chen, Feng-Jian Wang, Yung-Lin Chen. Proceedings of the 4th Asia-Pacific Software Engineering Computer Science Conference (APSEC '97 / ICSC '97).
\end{thebibliography}


\end{document}