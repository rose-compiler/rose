\chapter{The Copy Mechanism}

% TODO: sections on basic usage

\section{Advanced Usage}

The following sections describe some advanced usages of the copy
mechanism, demonstrating its flexibility.  This is achieved by subclassing
the {\tt SgCopyHelp} class.

The method we need to override is called {\tt copyAst}.  This method
is used by the copy mechanism to define the path of the traversal
and consequently the structure of the created AST.  This is achieved
by calling the {\tt copyAst} method for each attribute to be copied.
If this method makes a copy of the attribute using its own {\tt copy}
method, we achieve a deep copy via bottom-up traversal of the AST.
This is essentially what {\tt SgTreeCopy} does (with the exception of
types, which are shared pointers).

\section{Variable Substitution}

Say we want to create a version of a function where all instances of a
particular variable are replaced with a given expression.  An example of
an application of this is to a function containing a `for' loop such that
the bounds of the loop are fixed, the goal being to allow for certain
kinds of compiler optimization.

We proceed by creating a new subtype of the {\tt SgCopyHelp}
class, overriding the {\tt copyAst} method.  This method,
when encountering a reference to the specified variable, will
substitute it for a copy of the expression stored in its attribute.
Figure~\ref{Tutorial:example_varSubstCopy} shows how to implement a
translator based on this class.

\begin{figure}[!h]
{\indent
{\mySmallFontSize

% Do this when processing latex to generate non-html (not using latex2html)
\begin{latexonly}
   \lstinputlisting{\TutorialExampleDirectory/varSubstCopy.C}
\end{latexonly}

% Do this when processing latex to build html (using latex2html)
\begin{htmlonly}
   \verbatiminput{\TutorialExampleDirectory/varSubstCopy.C}
\end{htmlonly}

% end of scope in font size
}
% End of scope in indentation
}
\caption{Example source code for variable-substitution copy class.}
\label{Tutorial:example_varSubstCopy}
\end{figure}

Using the input program in Figure~\ref{Tutorial:exampleInputCode_varSubstCopy}
the translator processes the code and generates the output in 
Figure~\ref{Tutorial:exampleOutput_varSubstCopy}.

\begin{figure}[!h]
{\indent
{\mySmallFontSize

% Do this when processing latex to generate non-html (not using latex2html)
\begin{latexonly}
   \lstinputlisting{\TutorialExampleDirectory/inputCode_varSubstCopy.C}
\end{latexonly}

% Do this when processing latex to build html (using latex2html)
\begin{htmlonly}
   \verbatiminput{\TutorialExampleDirectory/inputCode_varSubstCopy.C}
\end{htmlonly}

% end of scope in font size
}
% End of scope in indentation
}
\caption{Example source code used as input to varSubstCopy.C}
\label{Tutorial:exampleInputCode_varSubstCopy}
\end{figure}

\begin{figure}[!h]
{\indent
{\mySmallFontSize

% Do this when processing latex to generate non-html (not using latex2html)
\begin{latexonly}
   \lstinputlisting{\TutorialExampleBuildDirectory/rose_inputCode_varSubstCopy.C}
\end{latexonly}

% Do this when processing latex to build html (using latex2html)
\begin{htmlonly}
   \verbatiminput{\TutorialExampleBuildDirectory/rose_inputCode_varSubstCopy.C}
\end{htmlonly}

% end of scope in font size
}
% End of scope in indentation
}
\caption{Output of input to varSubstCopy.C}
\label{Tutorial:exampleOutput_varSubstCopy}
\end{figure}

% An example of how this class may be applied is shown in Figure~\ref{Tutorial:copyHelp_inputSgVarSubstCopy} where the function `exampleFunction' is duplicated to the new function `exampleFunction20' where all references to its first parameter are substituted for the integer value 20.  Figure~\ref{Tutorial:copyHelp_outputSgVarSubstCopy} 

\section{Attributes}

The copy mechanism supports three kinds of attributes during a traversal: inherited, synthesised and accumulator (similar to the standard traversal mechanisms within ROSE).  Figure~\ref{Tutorial:attributeCopyExamples} shows example source code for all three types.

\begin{figure}[!h]
{\indent
{\mySmallFontSize

% Do this when processing latex to generate non-html (not using latex2html)
\begin{latexonly}
   \lstinputlisting{\TutorialExampleDirectory/attributeCopyExamples.C}
\end{latexonly}

% Do this when processing latex to build html (using latex2html)
\begin{htmlonly}
   \verbatiminput{\TutorialExampleDirectory/attributeCopyExamples.C}
\end{htmlonly}

% end of scope in font size
}
% End of scope in indentation
}
\caption{Example source code for variable-substitution copy class.}
\label{Tutorial:attributeCopyExamples}
\end{figure}
