\chapter{Dwarf Debug Support}

This chapter presents the support in ROSE for Dwarf debug information; its 
represnetation in the AST and its use in binary analysis tools.

In the following sections we use a small example (see figure \ref{Tutorial:dwarfAnalysisExampleSourceCode}) 
that demonstrates various features of Dwarf. The source code of our binary example is:

\begin{figure}[!h]
{\indent
{\mySmallFontSize

% Do this when processing latex to generate non-html (not using latex2html)
\begin{latexonly}
   \lstinputlisting{\TutorialExampleDirectory/dwarfAnalysis.C}
\end{latexonly}

% Do this when processing latex to build html (using latex2html)
\begin{htmlonly}
   \verbatiminput{\TutorialExampleDirectory/dwarfAnalysis.C}
\end{htmlonly}

% end of scope in font size
}
% End of scope in indentation
}
\caption{Example source code (typical for reading in a binary or source file).}
\label{Tutorial:dwarfAnalysisExampleSourceCode}
\end{figure}

Much larger binaries can be analized, but such larger binary executables are more
difficult to present (in this tutorial).


\section{ROSE AST of Dwarf IR nodes}

   ROSE tools process the binary into an AST that is used to represent all the
information in the binary executable.  Figure \ref{Tutorial:Dwarf_AST_example}
shows the subset of that AST (which includes the rest of the binary file format
and the disassembled instructions) that is specific to Dwarf.  A command line option
({\em -rose:visualize\_dwarf\_only}) is used to restrict the generated dot file 
visualization to just the Dwarf infomation.

\begin{figure}[h]
{\indent
{\mySmallFontSize

% Do this when processing latex to generate non-html (not using latex2html)
\begin{latexonly}
%  \includegraphics[scale=0.9]{\TutorialExampleBuildDirectory/inputCode_dwarfAnalysis}
   \includegraphics[viewport=5 1 50 500,scale=0.9]{\TutorialExampleBuildDirectory/inputCode_dwarfAnalysis}
\end{latexonly}

% Do this when processing latex to build html (using latex2html)
\begin{htmlonly}
   \includegraphics{\TutorialExampleBuildDirectory/inputCode_dwarfAnalysis}
\end{htmlonly}

% end of scope in font size
}
% End of scope in indentation
}
\caption{Dwarf AST (subset of ROSE binary AST).}
\label{Tutorial:Dwarf_AST_example}
\end{figure}



\section{Source Position to Instruction Address Mapping}

   One of the valuable parts of Dwarf is the mapping between the 
source lines and the instruction addresses at a statement level
(provided in the {\em .debug\_line} section).
Even though Dwarf does not represent all statements in the source
code, the mapping is significantly finer grainularity than that
provided at only the function level by the symbol table (which
identifies the functions by instruction address, but not the 
source file line numbers; the later requires source code analysis).

The example code in \ref{Tutorial:dwarfInstructionAddressToSourceLineExampleSourceCode}
shows the mapping between the source code lines and the instruction addresses.

\fixme{This need to be improved to correctly return sets of lines and sets of addresses in
       what might be a second interface.}

\begin{figure}[!h]
{\indent
{\mySmallFontSize

% Do this when processing latex to generate non-html (not using latex2html)
\begin{latexonly}
   \lstinputlisting{\TutorialExampleDirectory/dwarfInstructionAddressToSourceLineAnalysis.C}
\end{latexonly}

% Do this when processing latex to build html (using latex2html)
\begin{htmlonly}
   \verbatiminput{\TutorialExampleDirectory/dwarfInstructionAddressToSourceLineAnalysis.C}
\end{htmlonly}

% end of scope in font size
}
% End of scope in indentation
}
\caption{Example source code (typical for reading in a binary or source file).}
\label{Tutorial:dwarfInstructionAddressToSourceLineExampleSourceCode}
\end{figure}

Output from the compilation of this test code and running it with the example input
results in the output shown in figure \ref{Tutorial:dwarfInstructionAddressToSourceLineExampleOutput}.
This output shows the binary exectuables instruction address range (binary compiled on
Linux x86 system), the range of lines of source code used by the binary executable,
the mapping of a source code range of line numbers to the instruction addresses,
and the mapping of a range of instruction addresses to the source code line numbers.

%\fixme{Need to handle case of were each query can return a set of line or set of
%       addresses.  This may be a different interface for simplicty.}

\begin{figure}[!h]
{\indent
{\mySmallFontSize

% Do this when processing latex to generate non-html (not using latex2html)
\begin{latexonly}
   \lstinputlisting{\TutorialExampleBuildDirectory/inputCode_dwarfAnalysis.out}
\end{latexonly}

% Do this when processing latex to build html (using latex2html)
\begin{htmlonly}
   \verbatiminput{\TutorialExampleBuildDirectory/inputCode_dwarfAnalysis.out}
\end{htmlonly}

% end of scope in font size
}
% End of scope in indentation
}
\caption{Example source code (typical for reading in a binary or source file).}
\label{Tutorial:dwarfInstructionAddressToSourceLineExampleOutput}
\end{figure}






