\chapter{Abstract Handles to Language Constructs}
This chapter describes a reference design and its corresponding implementation for supporting abstract handles to language constructs in source code and optimization phases. 
It can be used to facilitate the interoperability between compilers and tools. 

The idea is to define unique identifiers for statements, loops, functions, 
and other language constructs in source code. Given the diverse user
requirements, an ideal specification should include multiple forms to specify a language construct. 
Current, we are interested in the following forms for specifying language constructs:
\begin{itemize}
\item Source file position information including path, filename, line and column number etc. 
GNU standard source position from
\url{http://www.gnu.org/prep/standards/html\_node/Errors.html} presents
some examples.  
\item Global or local numbering of specified language construct in source file
  (e.g. 2nd "do" loop in a global scope).  The file is itself specified using an abstract
          handle (typically generated from the file name). 
\item Global or local names of constructs. Some language constructs, such
as files, function definitions and namespace, have names which can be
used as their handle within a context.
\item Language-specific label mechanisms. They include named constructs in Fortran, numbered labels in Fortran, and statement labels in C and C++ and so on. 
\end{itemize}
In addition to human-readable forms, compilers and tools can generate
internal IDs for language constructs. It is up to compiler/tool developers
to provide a way to convert their internal representations into human-readable formats. 

We define an abstract handle as a unique representation for a language construct. It can have any of the human-readable or machine-generated forms. 
A handle can be used alone or combined with other handles to specify a language construct. 
A handle can also be converted from one form to another.
Abstract handles can have different lifetimes depending on their use and implementation. 
An abstract handle might be required to be persistent if it is used to reference a language construct that would be optimized over  multiple executions of one or more different tools. 
Where as an abstract-handle might be internally generated only for purposes of
           optimizations used in a single execution (e.g. optimization within a compiler). 

\section{Syntax}
A possible specification of language handles can have the following syntax:

\begin{verbatim}
/* a handle is a single handle item or a link of them separated by ::, or
other delimiters */
handle ::= handle_item | handle '::' handle_item

/* Each handle item consists of construct_type and a specifier. 
Or it can be a compiler generated id of any forms. */

handle_item ::= construct_type specifier | compiler_generated_handle

/* 
Construct types are implementation dependent.
An implementation can support a subset of legal constructs or all of them.
We define a minimum set of common construct type names here and 
will grow this list as needed.
*/
construct_type ::= Project|SourceFile|FunctionDeclaration|ForStatement|...

/* A specifier is used to locate a particular construct
  e.g: <name, "foo">
*/

specifier::= '<' specifier_type ',' specifier_value '>'                

/* tokens for specifier types could be name, position,numbering, label, etc. 
specifier type is necessary to avoid ambiguity for specifier values, 
because a same value could be interpreted in different specifier types otherwise
*/

specifier_type::= name | position | numbering | label 

/* Possible values for a specifier */

specifier_value::= string_lit|int_lit|position_value| label_value

/*A label could be either integer or string */
label_value::= int_lit | string_lit

/* Start and end source line and column information 
e.g.: 13.5-55.4,  13,  13.5 , 13.5-55 */
position_value:: = line_number[ '.' column_number][ '-' line_number[ '.' column_number]]

/* Integer value: one or more digits */
int_lit ::= [0-9]+

/* String value: start with a letter, followed by zero or more letters or digits */
string_lit ::= [a-z][a-z0-9]*

\end{verbatim}

\section{Examples}
We give some examples of language handles using the syntax mentioned above. 
ROSE AST's node type names are used as the construct type names. 
Other implementations can use their own construct type names.

\begin{itemize}
\item A file handle consisting of only one handle item: 
\begin{verbatim}
SourceFile<name,"/home/PERI/test111.f">
\end{verbatim}

\item A function handle using a named handle item, combined with a parent handle using a name also: 

\begin{verbatim}
SourceFile<name,"/home/PERI/test111.f">::FunctionDeclaration<name,"foo">
\end{verbatim}

\item A function handle using source position(A function starting at line 12, column 1 till line 30, column 1 within a file): 

\begin{verbatim}
SourceFile<name,"/home/PERI/test111.f">::FunctionDeclaration<position,"12.1-30.1">

\end{verbatim}
\item A function handle using numbering(The first function definition in a file): 
\begin{verbatim}
SourceFile<name,/home/PERI/test111.f">::FunctionDeclaration<numbering,1>
\end{verbatim}
\item A return statement using source position (A return statement at line 100):
\begin{verbatim}
SourceFile<name,/home/PERI/test222.c>::ReturnStatement<position,"100">

\end{verbatim}
\item A loop using numbering information (The second loop in function
main()): 
\begin{verbatim}
SourceFile<name,"/home/PERI/test222.c">::FunctionDeclaration<name,"main">::
ForStatement<numbering,2>
\end{verbatim}
\item A nested loop using numbering information (The first loop inside the second loop in function
main()): 
\begin{verbatim}
SourceFile<name,"/home/PERI/test222.c">::FunctionDeclaration<name,"main">::
ForStatement<numbering,2>::ForStatement<numbering,1>
\end{verbatim}

\end{itemize}

\section{Reference Implementation}
We provide a reference implementation of the abstract handle concept. 
The source files are located in \textit{src/midend/abstractHandle}.
A generic interface (abstract\_handle.h and abstract\_handle.cpp) provides
data structures and operations for manipulating abstract handles using source file positions, numbering, or names. 
Any compilers and tools can have their own implementations using the same interface.  
\subsection{Connecting to ROSE}
A ROSE adapter (roseAdapter.h and roseAdapter.cpp) using the interface is
provided as a concrete implementation for the maximum capability of the
implementation.

Figure~\ref{Tutorial:abstractHandle1} shows the code to generate abstract
handles for loops in an input source file (as in
Figure~\ref{Tutorial:abstractHandle1input}). 
Abstract handle constructors generate handles from abstract nodes, which are implemented using ROSE AST nodes. 
Source position is used by default to generate a handle item. 
Names or numbering are used instead when source position information is not available. 
The Constructor can also be used to generate a handle item using a
specified handle type (numbering handles in the example).
Figure~\ref{Tutorial:abstractHandle1out} is the output showing the generated handles for the loops.
%---------------------example 1. ----------------------
\begin{figure}[!h]
{\indent
{\mySmallestFontSize
% Do this when processing latex to generate non-html (not using latex2html)
\begin{latexonly}
  \lstinputlisting{\TutorialExampleDirectory/abstractHandle1.cpp}
\end{latexonly}

% Do this when processing latex to build html (using latex2html)
\begin{htmlonly}
   \verbatiminput{\TutorialExampleDirectory/abstractHandle1.cpp}
\end{htmlonly}

% end of scope in font size
}
% End of scope in indentation
}
\caption{Generated handles for loops: using constructors with or without a specified handle type}
\label{Tutorial:abstractHandle1}
\end{figure}
%------------------ input------------------------------
\begin{figure}[!h]
{\indent
{\mySmallestFontSize
% Do this when processing latex to generate non-html (not using latex2html)
\begin{latexonly}
  \lstinputlisting{\TutorialExampleDirectory/inputCode_AbstractHandle1.cpp}
\end{latexonly}

% Do this when processing latex to build html (using latex2html)
\begin{htmlonly}
   \verbatiminput{\TutorialExampleDirectory/inputCode_AbstractHandle1.cpp}
\end{htmlonly}

% end of scope in font size
}
% End of scope in indentation
}
\caption{Source code with some loops}
\label{Tutorial:abstractHandle1input}
\end{figure}


%---------------------example 1's output ----------------------
\begin{figure}[!h]
{\indent
{\mySmallestFontSize
% Do this when processing latex to generate non-html (not using latex2html)
\begin{latexonly}
  \lstinputlisting{\TutorialExampleBuildDirectory/abstractHandle1.outx}
\end{latexonly}

% Do this when processing latex to build html (using latex2html)
\begin{htmlonly}
   \verbatiminput{\TutorialExampleBuildDirectory/abstractHandle1.outx}
\end{htmlonly}

% end of scope in font size
}
% End of scope in indentation
}
\caption{Handles generated for loops}
\label{Tutorial:abstractHandle1out}
\end{figure}

Another example (shown in Figure~\ref{Tutorial:abstractHandle2})
demonstrates how to create handles using user-specified strings
representing handle items for language constructs within a source file
(shown in Figure~\ref{Tutorial:abstractHandle2input}). 
This is particularly useful to grab internal language constructs from handles provided by external software tools. 
The output of the example is given in Figure~\ref{Tutorial:abstractHandle2out}.

%---------------------example 2. ----------------------
\begin{figure}[!h]
{\indent
{\mySmallestFontSize
% Do this when processing latex to generate non-html (not using latex2html)
\begin{latexonly}
  \lstinputlisting{\TutorialExampleDirectory/abstractHandle2.cpp}
\end{latexonly}

% Do this when processing latex to build html (using latex2html)
\begin{htmlonly}
   \verbatiminput{\TutorialExampleDirectory/abstractHandle2.cpp}
\end{htmlonly}

% end of scope in font size
}
% End of scope in indentation
}
\caption{Generated handles from strings representing handle items}
\label{Tutorial:abstractHandle2}
\end{figure}

%------------------ input------------------------------
\begin{figure}[!h]
{\indent
{\mySmallestFontSize
% Do this when processing latex to generate non-html (not using latex2html)
\begin{latexonly}
  \lstinputlisting{\TutorialExampleDirectory/inputCode_AbstractHandle2.cpp}
\end{latexonly}

% Do this when processing latex to build html (using latex2html)
\begin{htmlonly}
   \verbatiminput{\TutorialExampleDirectory/inputCode_AbstractHandle2.cpp}
\end{htmlonly}

% end of scope in font size
}
% End of scope in indentation
}
\caption{Source code with some language constructs}
\label{Tutorial:abstractHandle2input}
\end{figure}


%---------------------example 2's output ----------------------
\begin{figure}[!h]
{\indent
{\mySmallestFontSize
% Do this when processing latex to generate non-html (not using latex2html)
\begin{latexonly}
  \lstinputlisting{\TutorialExampleBuildDirectory/abstractHandle2.outx}
\end{latexonly}

% Do this when processing latex to build html (using latex2html)
\begin{htmlonly}
   \verbatiminput{\TutorialExampleBuildDirectory/abstractHandle2.outx}
\end{htmlonly}

% end of scope in font size
}
% End of scope in indentation
}
\caption{Handles generated from string and their language constructs}
\label{Tutorial:abstractHandle2out}
\end{figure}

\subsection{Connecting to External Tools}
We give yet another example to demonstrate how to use the abstract interface
with any other tools, which may have less features in terms of
supported language constructs and their correlations compared to a compiler. 
Assume a tool operating on some simple for-loops
within a source file. The data structure representing such loops is given
in Figure~\ref{Tutorial:myloop}.

\begin{figure}[!h]
{\indent
{\mySmallestFontSize
% Do this when processing latex to generate non-html (not using latex2html)
\begin{latexonly}
  \lstinputlisting{\TutorialExampleDirectory/../src/midend/abstractHandle/myloop.h}
\end{latexonly}

% Do this when processing latex to build html (using latex2html)
\begin{htmlonly}
   \verbatiminput{\TutorialExampleDirectory/../src/midend/abstractHandle/myloop.h}
\end{htmlonly}

% end of scope in font size
}
% End of scope in indentation
}
\caption{A simple data structure representing a loop}
\label{Tutorial:myloop}
\end{figure}


An adapter (loopAdapter.h and loopAdapter.cpp) using the proposed abstract
handle interface is given in \textit{src/midend/abstractHandle}.
It provides a concrete implementation for the interface for the simple loops and adds a
node to support file nodes (Compared to the full-featured ROSE IR, this
additional file node is additional work when using tools without internal nodes
supporting files). 
The test program is given in Figure ~\ref{Tutorial:testMyLoop}.
Again, it creates a top level file handle first. Then a loop handle
(loop\_handle1) is
created within the file handle using its relative numbering information. 
loop\_handl2 is created from from its string format using file position
information. loop\_handl3 uses its relative numbering information within
loop\_handle1. 
\begin{figure}[!h]
{\indent
{\mySmallestFontSize
% Do this when processing latex to generate non-html (not using latex2html)
\begin{latexonly}
  \lstinputlisting{\TutorialExampleDirectory/../src/midend/abstractHandle/testMyLoop.cpp}
\end{latexonly}

% Do this when processing latex to build html (using latex2html)
\begin{htmlonly}
   \verbatiminput{\TutorialExampleDirectory/../src/midend/abstractHandle/testMyLoop.cpp}
\end{htmlonly}

% end of scope in font size
}
% End of scope in indentation
}
\caption{A test program for simple loops' abstract handles}
\label{Tutorial:testMyLoop}
\end{figure}


The output of the program is shown in Figure~\ref{Tutorial:testMyLoopOutput}
%\begin{verbatim}
\begin{figure}[!h]
{\indent
{\mySmallestFontSize
\begin{latexonly}
\begin{lstlisting} 
bash-3.00$ ./testMyLoop
Created a file handle:
SourceFile<name,file1.c>
Created a loop handle:
SourceFile<name,file1.c>::ForStatement<numbering,1>
Created a loop handle:
SourceFile<name,file1.c>::ForStatement<position,12>
Created a loop handle:
SourceFile<name,file1.c>::ForStatement<numbering,1>::ForStatement<numbering,1>
\end{lstlisting}
\end{latexonly}
% end of scope in font size
}
% End of scope in indentation
}
\caption{Output of the test program for simple loops' abstract handles}
\label{Tutorial:testMyLoopOutput}
\end{figure}


%\end{verbatim}

