\chapter{Partial Redundancy Elimination (PRE)}

   Figure~\ref{Tutorial:example_partialRedundancyElimination} shows an
example of how to call the Partial Redundancy Elimination (PRE) 
implemented by Jeremiah Willcock.  This transformation is useful for cleaning up code
generated from other transformations (used in Qing's loop optimizations).

\section{Source Code for example using PRE}

    Figure~\ref{Tutorial:example_partialRedundancyElimination}
shows an example translator which calls the PRE mechanism.

The input code is shown in figure~\ref{Tutorial:exampleInputCode_partialRedundancyElimination},
the output of this code is shown in 
figure~\ref{Tutorial:exampleOutput_partialRedundancyElimination}.

\begin{figure}[!h]
{\indent
{\mySmallFontSize

% Do this when processing latex to generate non-html (not using latex2html)
\begin{latexonly}
   \lstinputlisting{\TutorialExampleDirectory/partialRedundancyElimination.C}
\end{latexonly}

% Do this when processing latex to build html (using latex2html)
\begin{htmlonly}
   \verbatiminput{\TutorialExampleDirectory/partialRedundancyElimination.C}
\end{htmlonly}

% end of scope in font size
}
% End of scope in indentation
}
\caption{Example source code showing how use Partial Redundancy Elimination (PRE). }
\label{Tutorial:example_partialRedundancyElimination}
\end{figure}



\section{Input to Example Demonstrating PRE}

   Figure~\ref{Tutorial:exampleInputCode_partialRedundancyElimination}
shows the example input used for demonstration of Partial Redundancy 
Elimination (PRE) transformation.

\begin{figure}[!h]
{\indent
{\mySmallFontSize

% Do this when processing latex to generate non-html (not using latex2html)
\begin{latexonly}
   \lstinputlisting{\TutorialExampleDirectory/inputCode_partialRedundancyElimination.C}
\end{latexonly}

% Do this when processing latex to build html (using latex2html)
\begin{htmlonly}
   \verbatiminput{\TutorialExampleDirectory/inputCode_partialRedundancyElimination.C}
\end{htmlonly}

% end of scope in font size
}
% End of scope in indentation
}
\caption{Example source code used as input to program to the Partial Redundancy Elimination
    (PRE) transformation.}
\label{Tutorial:exampleInputCode_partialRedundancyElimination}
\end{figure}



\section{Final Code After PRE Transformation}

   Figure~\ref{Tutorial:exampleOutput_partialRedundancyElimination} 
shows the results from the use of PRE on an the example input code.


\begin{figure}[!h]
{\indent
{\mySmallFontSize

% Do this when processing latex to generate non-html (not using latex2html)
\begin{latexonly}
   \lstinputlisting{\TutorialExampleBuildDirectory/rose_inputCode_partialRedundancyElimination.C}
\end{latexonly}

% Do this when processing latex to build html (using latex2html)
\begin{htmlonly}
   \verbatiminput{\TutorialExampleBuildDirectory/rose_inputCode_partialRedundancyElimination.C}
\end{htmlonly}

% end of scope in font size
}
% End of scope in indentation
}
\caption{Output of input code after Partial Redundancy Elimination (PRE) transformation.}
\label{Tutorial:exampleOutput_partialRedundancyElimination}
\end{figure}



























